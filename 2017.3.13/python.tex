% !TeX encoding = UTF-8
老师说的方法在cmd命令行模式和Python交互环境下都出错,出错信息是syntax error。我自己试了好过次,正确的方法是:在cmd命令行模式下输入:
E:\python27\hello.py

E:\python27是我的存储路径。

python E:\python27\hello.py

还有同学问,能不能像.exe文件那样直接运行.py文件呢?在Windows上是不行的.
\section{Print}
print语句也可以跟上多个字符串,用逗号“,”隔开,就可以连成一串输出.
文字用单引号或者双引号括起来.
\section{raw_input()}
从raw_input()读取的内容永远以字符串的形式返回,把字符串和整数比较就不会得到期待的结果,必须先用int()把字符串转换为我们想要的整型:
\section{}
以#开头的语句是注释,注释是给人看的,可以是任意内容,解释器会忽略掉注释。其他每一行都是一个语句,当语句以冒号“:”结尾时,缩进的语句视为代码块。

缩进有利有弊。好处是强迫你写出格式化的代码,但没有规定缩进是几个空格还是Tab。按照约定俗成的管理,应该始终坚持使用4个空格的缩进。

缩进的另一个好处是强迫你写出缩进较少的代码,你会倾向于把一段很长的代码拆分成若干函数,从而得到缩进较少的代码。

缩进的坏处就是“复制-粘贴”功能失效了,这是最坑爹的地方。当你重构代码时,粘贴过去的代码必须重新检查缩进是否正确。此外,IDE很难像格式化Java代码那样格式化Python代码。

最后,请务必注意,\textbf{Python程序是大小写敏感的,如果写错了大小写,程序会报错}。
\chapter{数据类型和变量}
\section{整数}
Python可以处理任意大小的整数,当然包括负整数,在程序中的表示方法和数学上的写法一模一样,例如:1,100,-8080,0,等等。

计算机由于使用二进制,所以,有时候用十六进制表示整数比较方便,十六进制用0x前缀和0-9,a-f表示,例如:0xff00,0xa5b4c3d2,等等。
\section{浮点数}
浮点数也就是小数,之所以称为浮点数,是因为按照科学记数法表示时,一个浮点数的小数点位置是可变的,比如,1.23x109和12.3x108是相等的。浮点数可以用数学写法,如1.23,3.14,-9.01,等等。但是对于很大或很小的浮点数,就必须用科学计数法表示,把10用e替代,1.23x109就是1.23e9,或者12.3e8,0.000012可以写成1.2e-5,等等。

整数和浮点数在计算机内部存储的方式是不同的,整数运算永远是精确的(除法难道也是精确的?是的!),而浮点数运算则可能会有四舍五入的误差。
\section{字符串}
字符串是以''或""括起来的任意文本,比如'abc',"xyz"等等。请注意,''或""本身只是一种表示方式,不是字符串的一部分,因此,字符串'abc'只有a,b,c这3个字符。如果'本身也是一个字符,那就可以用""括起来,比如"I'm OK"包含的字符是I,',m,空格,O,K这6个字符。

如果字符串内部既包含'又包含"怎么办?可以用转义字符\来标识,比如:

'I\'m \"OK\"!'

表示的字符串内容是:

I'm "OK"!

转义字符\可以转义很多字符,比如\n表示换行,\t表示制表符,字符\本身也要转义,所以\\表示的字符就是\,可以在Python的交互式命令行用print打印字符串看看:

>>> print 'I\'m ok.'
I'm ok.
>>> print 'I\'m learning\nPython.'
I'm learning
Python.
>>> print '\\\n\\'
\
\

如果字符串里面有很多字符都需要转义,就需要加很多\,为了简化,Python还允许用r''表示''内部的字符串默认不转义,可以自己试试:

>>> print '\\\t\\'
\       \
>>> print r'\\\t\\'
\\\t\\

如果字符串内部有很多换行,用\n写在一行里不好阅读,为了简化,Python允许用'''...'''的格式表示多行内容,可以自己试试:

>>> print '''line1
... line2
... line3'''
line1
line2
line3

上面是在交互式命令行内输入,如果写成程序,就是:

print '''line1
line2
line3'''

多行字符串'''...'''还可以在前面加上r使用,请自行测试。
\section{布尔值}
布尔值和布尔代数的表示完全一致,一个布尔值只有True、False两种值,要么是True,要么是False,在Python中,可以直接用True、False表示布尔值(请注意大小写),也可以通过布尔运算计算出来:

>>> True
True
>>> False
False
>>> 3 > 2
True
>>> 3 > 5
False

布尔值可以用and、or和not运算。

and运算是与运算,只有所有都为True,and运算结果才是True:

>>> True and True
True
>>> True and False
False
>>> False and False
False

or运算是或运算,只要其中有一个为True,or运算结果就是True:

>>> True or True
True
>>> True or False
True
>>> False or False
False

not运算是非运算,它是一个单目运算符,把True变成False,False变成True:

>>> not True
False
>>> not False
True

布尔值经常用在条件判断中,比如:

if age >= 18:
    print 'adult'
else:
    print 'teenager'
\section{空值}
空值是Python里一个特殊的值,用None表示。None不能理解为0,因为0是有意义的,而None是一个特殊的空值。

此外,Python还提供了列表、字典等多种数据类型,还允许创建自定义数据类型,我们后面会继续讲到.
\chapter{编码}
搞清楚了ASCII、Unicode和UTF-8的关系,我们就可以总结一下现在计算机系统通用的字符编码工作方式:

在计算机内存中,统一使用Unicode编码,当需要保存到硬盘或者需要传输的时候,就转换为UTF-8编码。

用记事本编辑的时候,从文件读取的UTF-8字符被转换为Unicode字符到内存里,编辑完成后,保存的时候再把Unicode转换为UTF-8保存到文件:

浏览网页的时候,服务器会把动态生成的Unicode内容转换为UTF-8再传输到浏览器:

所以你看到很多网页的源码上会有类似<meta charset="UTF-8" />的信息,表示该网页正是用的UTF-8编码。

\section{Python的字符串}

\section{list[]}
Python内置的一种数据类型是列表:list。list是一种有序的集合,可以随时添加和删除其中的元素。

比如,列出班里所有同学的名字,就可以用一个list表示:

>>> classmates = ['Michael', 'Bob', 'Tracy']
>>> classmates
['Michael', 'Bob', 'Tracy']

变量classmates就是一个list。用len()函数可以获得list元素的个数:

>>> len(classmates)
3

用索引来访问list中每一个位置的元素,记得索引是从0开始的:

>>> classmates[0]
'Michael'

当索引超出了范围时,Python会报一个IndexError错误,所以,要确保索引不要越界,记得最后一个元素的索引是len(classmates) - 1。

如果要取最后一个元素,除了计算索引位置外,还可以用-1做索引,直接获取最后一个元素:

>>> classmates[-1]
'Tracy'

以此类推,可以获取倒数第2个、倒数第3个:

>>> classmates[-2]

当然,倒数第4个就越界了。

list是一个可变的有序表,所以,可以往list中追加元素到末尾:

>>> classmates.append('Adam')
>>> classmates
['Michael', 'Bob', 'Tracy', 'Adam']

也可以把元素插入到指定的位置,比如索引号为1的位置,insert:

>>> classmates.insert(1, 'Jack')
>>> classmates
['Michael', 'Jack', 'Bob', 'Tracy', 'Adam']

要删除list末尾的元素,用pop()方法:

>>> classmates.pop()
'Adam'
>>> classmates
['Michael', 'Jack', 'Bob', 'Tracy']

要删除指定位置的元素,用pop(i)方法,其中i是索引位置:

>>> classmates.pop(1)
'Jack'

要把某个元素替换成别的元素,可以直接赋值给对应的索引位置:

>>> classmates[1] = 'Sarah'
>>> classmates
['Michael', 'Sarah', 'Tracy']

list里面的元素的数据类型也可以不同,比如:

>>> L = ['Apple', 123, True]

list元素也可以是另一个list,比如:

>>> s = ['python', 'java', ['asp', 'php'], 'scheme']
>>> len(s)
4

要注意s只有4个元素,其中s[2]又是一个list,如果拆开写就更容易理解了:

>>> p = ['asp', 'php']
>>> s = ['python', 'java', p, 'scheme']

要拿到'php'可以写p[1]或者s[2][1],因此s可以看成是一个二维数组,类似的还有三维、四维……数组,不过很少用到。

如果一个list中一个元素也没有,就是一个空的list,它的长度为0:

>>> L = []
>>> len(L)
0
\section{tuple()}
\chapter{条件判断和循环}
注意不要少写了冒号:
只要x是非零数值、非空字符串、非空list等,就判断为True,否则为False。
\section{}
Python的循环有两种,一种是for...in循环,依次把list或tuple中的每个元素迭代出来,看例子:

names = ['Michael', 'Bob', 'Tracy']
for name in names:
    print name

执行这段代码,会依次打印names的每一个元素:

Michael
Bob
Tracy

所以for x in ...循环就是把每个元素代入变量x,然后执行缩进块的语句。

再比如我们想计算1-10的整数之和,可以用一个sum变量做累加:
第二种循环是while循环,只要条件满足,就不断循环,条件不满足时退出循环。比如我们要计算100以内所有奇数之和,可以用while循环实现:

sum = 0
n = 99
while n > 0:
    sum = sum + n
    n = n - 2
print sum

在循环内部变量n不断自减,直到变为-1时,不再满足while条件,循环退出。
\chapter{使用dict\{\}和set}

dict的key必须是不可变对象。
dict\{\};
set
\chapter{函数}
在Python中,定义一个函数要使用def语句,依次写出函数名、括号、括号中的参数和冒号:,然后,在缩进块中编写函数体,函数的返回值用return语句返回。
空函数

如果想定义一个什么事也不做的空函数,可以用pass语句:

def nop():
    pass
一是必选参数在前,默认参数在后,否则Python的解释器会报错(思考一下为什么默认参数不能放在必选参数前面);

二是如何设置默认参数。

当函数有多个参数时,把变化大的参数放前面,变化小的参数放后面。变化小的参数就可以作为默认参数。

使用默认参数有什么好处?最大的好处是能降低调用函数的难度。

所以,定义默认参数要牢记一点:默认参数必须指向不变对象!
\section{可变参数}
所以,我们把函数的参数改为可变参数:

def calc(*numbers):
    sum = 0
    for n in numbers:
        sum = sum + n * n
    return sum

定义可变参数和定义list或tuple参数相比,仅仅在参数前面加了一个*号。在函数内部,参数numbers接收到的是一个tuple,因此,函数代码完全不变。但是,调用该函数时,可以传入任意个参数,包括0个参数:

>>> calc(1, 2)
5
>>> calc()
0

如果已经有一个list或者tuple,要调用一个可变参数怎么办?Python允许你在list或tuple前面加一个*号,把list或tuple的元素变成可变参数传进去:

>>> nums = [1, 2, 3]
>>> calc(*nums)
14

这种写法相当有用,而且很常见。
\section{关键字参数}
可变参数允许你传入0个或任意个参数,这些可变参数在函数调用时自动组装为一个tuple。而关键字参数允许你传入0个或任意个含参数名的参数,这些关键字参数在函数内部自动组装为一个dict。请看示例:

def person(name, age, **kw):
    print 'name:', name, 'age:', age, 'other:', kw

函数person除了必选参数name和age外,还接受关键字参数kw。在调用该函数时,可以只传入必选参数:

>>> person('Michael', 30)
name: Michael age: 30 other: {}

也可以传入任意个数的关键字参数:

>>> person('Bob', 35, city='Beijing')
name: Bob age: 35 other: {'city': 'Beijing'}
>>> person('Adam', 45, gender='M', job='Engineer')
name: Adam age: 45 other: {'gender': 'M', 'job': 'Engineer'}
\section{参数组合}
参数定义的顺序必须是:必选参数、默认参数、可变参数和关键字参数。

def func(a, b, c=0, *args, **kw):
    print 'a =', a, 'b =', b, 'c =', c, 'args =', args, 'kw =', kw
    
最神奇的是通过一个tuple和dict,你也可以调用该函数:

>>> args = (1, 2, 3, 4)
>>> kw = {'x': 99}
>>> func(*args, **kw)
a = 1 b = 2 c = 3 args = (4,) kw = {'x': 99}

所以,对于任意函数,都可以通过类似func(*args, **kw)的形式调用它,无论它的参数是如何定义的。
\section{函数作为返回值}
内部函数sum可以引用外部函数lazy_sum的参数和局部变量,当lazy_sum返回函数sum时,相关参数和变量都保存在返回的函数中,这种称为“闭包(Closure)”的程序结构拥有极大的威力。
\section{装饰器}
函数对象有一个__name__属性,可以拿到函数的名字:

>>> now.__name__
'now'
>>> f.__name__
'now'
现在,假设我们要增强now()函数的功能,比如,在函数调用前后自动打印日志,但又不希望修改now()函数的定义,这种在代码运行期间动态增加功能的方式,称之为“装饰器”(Decorator)。

本质上,decorator就是一个返回函数的高阶函数。所以,我们要定义一个能打印日志的decorator,可以定义如下:

def log(func):
    def wrapper(*args, **kw):
        print 'call %s():' % func.__name__
        return func(*args, **kw)
    return wrapper

观察上面的log,因为它是一个decorator,所以接受一个函数作为参数,并返回一个函数。
\chapter{类}
\section{绑定方法}
给类绑定方法;
给实例绑定方法;
\section{slot}
白名单?
\section{链式调用}
很简单,返回对象自己就行了,即return self.

废话不多说,上代码。

[python] view plain copy

    class Person:  
        def name(self, name):  
            self.name = name  
            return self  
      
        def age(self, age):  
            self.age = age  
            return self  
      
        def show(self):  
            print "My name is", self.name, "and I am", self.age, "years old."  
      
    p = Person()  
    p.name("Li Lei").age(15).show()  
\section{super函数}
\section{元类}
\chapter{字符串处理}
.replace('','')方法
正则提取方法
.find()
\chapter{文件写入}
f=file('','wb')
\\需要转义
f.write()
f.close()
\section{文件夹}
os模块
os.mkdir
os.makedirs
http://www.pythontab.com/html/2014/pythonjichu_0121/681.html
\chapter{命令行调用参数}
脚本名:    sys.argv[0]
在sys.argv[i]内